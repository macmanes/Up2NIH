%\documentclass[11pt]{article}
%\usepackage{framed, color}
%\usepackage{textpos}
%\usepackage{natbib}
%\usepackage{geometry}
%\usepackage[hidelinks]{hyperref}
%\usepackage{textcomp}
%\usepackage{graphicx}
%\usepackage{fancybox}
%\usepackage{setspace}
%\hypersetup{colorlinks=false, urlcolor=blue, citecolor=black}
%\usepackage{soul}
%\usepackage{geometry}
%\usepackage{fancyhdr}
%\usepackage{wrapfig}
%\usepackage{mdframed}
%\usepackage{fontspec}
%
%\renewcommand\refname{Bibliography and References Cited}
%\newgeometry{top=.5in, bottom=.5in, left=.5in, right=.5in}
%\setmainfont{Arial}
%\linespread{1.2}
%\urlstyle{same}
%\setlength{\parindent}{1cm}
%
%\begin{document}
%
%%\setcounter{page}{0}
%
%%\fancyhead[CO]{Matthew D. MacManes | Specific Aims}
%%\pagestyle{fancy}
%%\setcounter{page}{1}
%\pagestyle{empty}
%%\raggedright

\noindent \textbf{SPECIFIC AIMS}

The maintenance of water balance is critical for survival. Humans are exquisitely sensitive to changes in hydration status, with slight derangement eliciting physiologic compromise. When the loss of water exceeds dietary intake, dehydration - and in extreme cases, death - can occur. Though providing drinking water is ultimately curative, this is not always possible (\textit{e.g.,} illness, water contamination, natural disasters, combat soldiers), and as a consequence, millions of people die every year as a direct result of dehydration, with countless others suffering physiologic and cognitive impairment. While decades of study in humans has elucidated the pathophysiology related to dehydration, that no current model can survive despite severe and prolonged dehydration represents a critically important gap in our current approach. In contrast to humans, animals living in desert habitats thrive without water and endure extreme heat and intense drought, as a direct result of unique adaptations. These adaptations allow them to survive conditions fatal to humans and most other animals. Despite being a well-known phenomenon with obvious implications for human health, we know very little of the underlying mechanisms that allow for survival in deserts. \textbf{The proposed research uses a novel desert-adapted dehydration-tolerant rodent model and an innovative approach integrating physiology, evolutionary genomics, and computational biology to understand how animals survive despite severe dehydration.} Indeed, this model offers the scientific community a unique opportunity to gain a deep understanding into the physiology and genomics of osmoregulation in extreme environments – an important insight that is impossible the achieve using a traditional model system like \textit{Mus} that, like humans, die when subjected to these conditions.

This project lays the groundwork for our \ul{\emph{long-term research goal}} – to identify the causal links between desert adapted animals ability to survive despite dehydration and the patterns of gene expression, methylation, and allelic variation. To achieve these, our \ul{\emph{specific aims}} of this project are: \\

\noindent \textbf{(1) To characterize the physiologic response to extreme water restriction and heat.} The working hypothesis is that desert survival is enabled by limiting water loss in the urine, feces, and respiratory tract via modifications to genitourinary, gastrointestinal, and respiratory physiology relative to model organisms like \textit{Mus}. This hypothesis will be tested via environmental manipulations coupled with careful measurement of physiological response.

\noindent \textbf{(2) To characterize the genomic response (differential gene expression, patterns of methylation or isoform use) to extreme water restriction and heat.} The working hypothesis is that while desert-adapted mice may demonstrate genome wide expression patterns suggestive of stress (e.g. activation of heat shock protein, vasopressin responsive pathways) during dehydration, these responses function to preserve normal physiology and thus serum chemistry will be similar to mice with unrestricted access to water.

\noindent \textbf{(3) To determine the ontogeny of extreme osmoregulatory ability, from the neonatal period during which fluid (milk) intake is obligate through weaning, when oral fluid intake is exceptionally rare. } The hypothesis here is that patterns of renal gene expression during fetal development through weaning will resemble patterns of gene expression, isoform use, and methylation typical of adult mice when water is freely available. \\


\noindent The proposed project aims to integrate studies of physiology, genomics, and computational biology to gain a deep understanding of a fundamental physiological problem – how to conserve water when intake is limited. \ul{\emph{Although dehydration is both common and dangerous, the biology underlying its physiological effects is currently invisible to researchers using traditional mammalian models of disease that lack the eco-evolutionary history present in desert-adapted mice}}. This project will fill an important gap in our understanding, which is in support of the research aims of the National Institute of Diabetes and Digestive and Kidney Diseases (NIDDK), and specifically, of the Kidney Basic Research program, which supports fundamental research on the normal development, structure, and function of the kidney.

%\end{document}
























