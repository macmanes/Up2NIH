%\documentclass[11pt]{article}
%\usepackage{framed, color}
%\usepackage{textpos}
%\usepackage{natbib}
%\usepackage{geometry}
%\usepackage[hidelinks]{hyperref}
%\usepackage{textcomp}
%\usepackage{graphicx}
%\usepackage{fancybox}
%\usepackage{setspace}
%\hypersetup{colorlinks=false, urlcolor=blue, citecolor=black}
%\usepackage{soul}
%\usepackage{geometry}
%\usepackage{fancyhdr}
%\usepackage{wrapfig}
%\usepackage{mdframed}
%\usepackage{fontspec}
%
%\renewcommand\refname{Bibliography and References Cited}
%\newgeometry{top=.5in, bottom=.5in, left=.5in, right=.5in}
%\setmainfont{Arial}
%\linespread{1.2}
%\urlstyle{same}
%\setlength{\parindent}{1cm}
%
%\begin{document}

\noindent \textbf{SIGNIFICANCE} \\

\noindent Dehydration, whether caused by exposure to extreme environmental conditions, water deprivation, or by infection (e.g. diarrheal illnesses) represents a significant threat to human life. In spite of modern medicine, millions of people die every year from dehydration. Compounding issues of exposure and illness are public health issues regarding the delivery of safe drinking water. With global climate change, these challenges are thought to become only more severe and as a result, \ul{research providing insight into the mechanisms underlying physiologic resistance to acute and chronic dehydration is urgently needed.} The response to acute dehydration in humans and traditional mammalian models is generally maladaptive and may include death - this response limits our ability to develop novel insights into this important cause of human mortality. As such, the study of dehydration-tolerant mammalian models will significantly enhance our understanding, and will provide fodder for novel treatments. \textbf{The proposed work aims to study extreme osmoregulation in a uniquely suited novel desert-adapted model organism.} \\

\noindent While the mechanisms underlying physiological compromise in dehydration are well characterized \citep{Roberts:2010fl}, some animals possess the ability, much unlike humans, to osmoregulate despite extreme heat and a complete lack of extrinsic water intake \citep{NAGY:1994ta}. Specifically, highly adapted desert mice may never drink water, produce an extremely viscous urine, or no urine at all, and excrete urea in the form of uric acid crystals in the feces \citep{SCHMIDTNIELSEN:1952wi}. This phenotype results in an animal that is very resistant to dehydration-related physiologic compromise, and is in stark contrast to the phenotype of humans and traditional model organisms (e.g. \textit{Mus} and \textit{Rattus}). Although model organisms are attractive targets for study, they lack the requisite biology which may limit insight. In contrast with traditional model organisms, non-model desert-adapted organisms may provide a unique opportunity to study dehydration tolerance, though they typically lack many of the genomic and physiologic tools characteristic of model organisms. Despite this, renal gene expression has been characterized for several genes in desert animals, and was shown to be highly derived in some (e.g. \textit{Dipodomys} \citep{Huang:2001ti}), but not in others (e.g. \textit{Notomys}: \cite{Weaver:1994wv}). No studies characterizing genome-wide patterns of gene expression, methylation or isoform use in desert-adapted water stressed animals have been done and therefore the extent to which differences in these parameters underlie phenotype remains unknown. \ul{The proposed work effectively integrates the power of a model organism with the unique biology of a desert-adapted rodent, the cactus mouse (\textit{Peromyscus eremicus}), to generate insights into extreme osmoregulation not current possible with existing mammalian models of dehydration.} \\


\noindent \textbf{INNOVATION} \\


\noindent The proposed work recognizes that successful treatment requires an appropriate model, and while traditional models are powerful, they lack the biology (extreme osmoregulation) upon which more successful interventions may be modeled. The desert-adapted rodent \textit{P. eremicus} retains many of the beneficial characteristics of model organisms, while enhancing opportunity to assay interesting biological phenomenon. In addition to this fundamental innovation, the project is innovative in a number of other ways including experimental, conceptual and technical innovation. The proposed project leverages unprecedented control over environmental conditions (\textit{e.g.,} a desert chamber) using an ideally suited novel model organism and unique analytical methods to understand the physiologic and genomic response to water deprivation.
 


%\end{document}
